\documentclass[twoside]{article}
\setlength{\oddsidemargin}{0.25 in}
\setlength{\evensidemargin}{-0.25 in}
\setlength{\topmargin}{-0.6 in}
\setlength{\textwidth}{6.5 in}
\setlength{\textheight}{8.5 in}
\setlength{\headsep}{0.75 in}
\setlength{\parindent}{0 in}
\setlength{\parskip}{0.1 in}

%
% ADD PACKAGES here:
%
\usepackage{amsmath,amsfonts,graphicx}
\usepackage[most]{tcolorbox}
\usepackage[colorlinks=true, linkcolor=red, urlcolor=blue]{hyperref}

\DeclareSymbolFont{extraup}{U}{zavm}{m}{n}
\DeclareMathSymbol{\varheart}{\mathalpha}{extraup}{86}
\DeclareMathSymbol{\vardiamond}{\mathalpha}{extraup}{87}

%
% The following commands set up the lecnum (lecture number)
% counter and make various numbering schemes work relative
% to the lecture number.
%
\newcounter{lecnum}
\renewcommand{\thepage}{\thelecnum-\arabic{page}}
\renewcommand{\thesection}{\thelecnum.\arabic{section}}
\renewcommand{\theequation}{\thelecnum.\arabic{equation}}
\renewcommand{\thefigure}{\thelecnum.\arabic{figure}}
\renewcommand{\thetable}{\thelecnum.\arabic{table}}

%
% The following macro is used to generate the header.
%
\newcommand{\lecture}[4]{
   \pagestyle{myheadings}
   \thispagestyle{plain}
   \newpage
   \setcounter{lecnum}{#1}
   \setcounter{page}{1}
   \noindent
   \begin{center}
   \framebox{
      \vbox{\vspace{2mm}
    \hbox to 6.28in { {\bf STAT 512: Statistical Inference
	\hfill Autumn 2025} }
       \vspace{4mm}
       \hbox to 6.28in { {\Large \hfill Lecture #1: #2  \hfill} }
       \vspace{2mm}
       \hbox to 6.28in { {\it Instructor: #3 \hfill Tony Lei} }
      \vspace{2mm}}
   }
   \end{center}
   \markboth{Lecture #1: #2}{Lecture #1: #2}

   \vspace*{4mm}
}

%for the motivation box
\tcbset{
  motivationstyle/.style={
    colback=gray!10!white,   % light gray background
    colframe=gray!70!black,  % darker gray border
    fonttitle=\bfseries,
    title=Motivation,
    sharp corners,
    boxrule=0.8pt,
    coltitle=black,
    enhanced
  }
}

\newenvironment{motivation}
  {\begin{tcolorbox}[motivationstyle]}
  {\end{tcolorbox}}

%Use this command for a figure
\newcommand{\fig}[3]{
			\vspace{#2}
			\begin{center}
			Figure \thelecnum.#1:~#3
			\end{center}
	}

% Red note command
\newcommand{\note}[1]{\textcolor{red}{#1}}

% Use these for theorems, lemmas, proofs, etc.
\newtheorem{theorem}{Theorem}[lecnum]
\newtheorem{lemma}[theorem]{Lemma}
\newtheorem{proposition}[theorem]{Proposition}
\newtheorem{claim}[theorem]{Claim}
\newtheorem{question}[theorem]{Question}
\newtheorem{answer}[theorem]{Answer}
\newtheorem{corollary}[theorem]{Corollary}
\newtheorem{definition}[theorem]{Definition}
\newenvironment{proof}{{\bf Proof:}}{\hfill\rule{2mm}{2mm}}

% **** ADDITIONAL MACROS:
\newcommand\E{\mathbb{E}}
\newcommand\F{\mathcal{F}}
\newcommand{\prob}{\mathbb{P}}

\begin{document}
\lecture{1}{Introduction to Probability and Statistics}{Ema Perkovic}

\section{Sample Space and Probability Measure}
First, let's recall some basic definitions of set theory and probability. 

The \textbf{sample space} ($\Omega$) is the set of all possible outcomes of a random experiment. A single outcome from the sample space is denoted as $\omega$. An \textbf{event} is a subset of the sample space, that is, $A\subseteq \Omega$. The \textbf{complement} of that event is denoted as $A^{c}$.

Let $A_i,A_j,\dots$ be events in $\Omega$. In general, we say two events are \textbf{pairwise disjoint} if  
$$A_i\cap A_j = \emptyset, \quad i\neq j.$$  
For a set $\Omega$, $A_1,\dots,A_k$ form a \textbf{partition} of $\Omega$ if  
$$\bigcup_{i=1}^kA_i = \Omega.$$
 
Now that we have the basics out of the way, let us define $\sigma$-algebras.

\subsection{$\sigma$-algebra}
\begin{motivation}
Given any subset of $\Omega$, we want to be able to define a \emph{measure} on it. Is every set measurable? Which ones are measurable? What is a measure?
\end{motivation}
\begin{definition}
    A \textbf{sigma algebra} $\mathcal{F}$ is a \textbf{collection of subsets} of $\Omega$ that satisfies:
\begin{enumerate}
  \item[(A1)] (entire/empty set) $\Omega \in \mathcal{F}$, $\emptyset \in \mathcal{F}$.
  \item[(A2)] (complement) If $A \in \mathcal{F}$, then $A^c \in \mathcal{F}$.
  \item[(A3)] (countable unions) If $A_1,A_2,\dots \in \mathcal{F}$, then $\bigcup_{i=1}^{\infty} A_i \in \mathcal{F}$.\\  
  \note{This is also true for finite unions.}\\
  \note{By De Morgan’s law, countable intersections are also in $\mathcal{F}$.}
  \end{enumerate}
\end{definition}

Now, given a pair $(\Omega,\mathcal{F})$, we say it is a \textbf{measurable space}. This means that a \textbf{measure} can be assigned to it.  
\note{Not to be confused with a \textbf{measure space}, which includes the measure itself.}

\begin{definition}
    A \textbf{measure} is a function $\mu:\mathcal{F}\rightarrow [0,\infty]$ (in general, could be signed $\mu:\mathcal{F}\rightarrow \mathbb{R}$) such that:
    \begin{itemize}
        \item $\mu(\emptyset)=0$.
        \item For any \textbf{pairwise disjoint} collection $\{A_i\}_{i=1}^{\infty}$,
        \[
            \mu\!\left(\bigcup_{i=1}^\infty A_i\right)=\sum_{i=1}^\infty \mu(A_i).
        \]
    \end{itemize}
\end{definition}

Given $\Omega$, there are many possible $\sigma$-algebras. Which one do we choose?  
The \textbf{Borel $\sigma$-algebra} on $\mathbb{R}$, denoted $\mathcal{B}(\mathbb{R})$, is the smallest $\sigma$-algebra containing all open intervals $(a,b)$ ($a,b\in\mathbb{R}$). Equivalently, it is generated by any of the families
$\{(a,b)\}$, $\{[a,b)\}$, or $\{(-\infty,a]\}$ for $a,b\in\mathbb{R}$.

We use the Borel $\sigma$-algebra because it contains all the “nice” sets we care about. In other words, it is the smallest $\sigma$-algebra that contains all the “events” that we want to measure.

\begin{definition}
    Given a measure space $(\Omega,\mathcal{F},\mu)$, it is a \textbf{probability space} if
    $$\mu(\Omega)=1.$$ 
    In this case, $\mu$ is called a \textbf{probability measure}, denoted by $\mathbb{P}$. It must satisfy:
    \begin{itemize}
        \item [(P1)] $\prob(\Omega) = 1$.
        \item [(P2)] $\prob(A)\geq 0, \quad \forall A\in \F$.
        \item [(P3)] For any \textbf{mutually exclusive events} $\{A_i\}_{i=1}^{\infty}$,
        \[
            \prob\!\left(\bigcup_{i=1}^\infty A_i\right)=\sum_{i=1}^\infty \prob(A_i).
        \]
    \end{itemize}
\end{definition}

The axioms (P1)–(P3) imply the following:

\begin{theorem}
For any $A,B\in\mathcal{F}$,
\begin{itemize}
  \item $\mathbb{P}(\emptyset)=0$;
  \item $0\le \mathbb{P}(A)\le 1$;
  \item $A\subseteq B \ \Rightarrow\ \mathbb{P}(A)\le \mathbb{P}(B)$;
  \item $\mathbb{P}(A^{c}) = 1-\mathbb{P}(A)$;
  \item $\mathbb{P}(A\cup B)=\mathbb{P}(A)+\mathbb{P}(B)-\mathbb{P}(A\cap B)$.
\end{itemize}
\end{theorem}

From (P3), we also get the following two properties:

\paragraph{Continuity from below.}
If $A_1\subseteq A_2\subseteq\cdots$ (an increasing sequence), then
\[
  \mathbb{P}\!\left(\bigcup_{n=1}^{\infty} A_n\right)
  = \lim_{n\to\infty}\mathbb{P}(A_n).
\]

\paragraph{Continuity from above.}
If $A_1\supseteq A_2\supseteq\cdots$ (a decreasing sequence), then
\[
  \mathbb{P}\!\left(\bigcap_{n=1}^{\infty} A_n\right)
  = \lim_{n\to\infty}\mathbb{P}(A_n).
\]

\end{document}